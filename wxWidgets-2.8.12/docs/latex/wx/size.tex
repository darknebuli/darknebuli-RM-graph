U2FsdGVkX19fTms3OENxUuiSyMIoebjgJfCbmHAHJ2PoksjCKHm44CXwm5hwBydj
6JLIwih5uOAl8JuYcAcnY+iSyMIoebjgJfCbmHAHJ2MRgK1O4VPG73jxQYlLm0F0
E5KS5QvjdHRLYjjspsMCVEmK68odVo29vHwgjbQwSkV7CWUMjTBXASfcreKGZogK
tYrpYe5DsIwqcClB1lyiXU1skKglvnSUtOoUR/hKe8gcXtxt3+Q2VC2y9JM4D8rO
JuZ050cTJhUcNEIEc75O+Ih615ZZAy3peyeHYWVdicW1hsWrxxvWrOCA644cY/3e
auMZ/X443gSVgiaQHUmsVW8rH6/UNLxzxdHbfj5Jbpd8Ipjeay64psO6gCddZWkQ
+kj/QHKeZrRiI8PNwz6cFhMX0tJAkmBCiBgM2ai1IWjoksjCKHm44CXwm5hwBydj
6JLIwih5uOAl8JuYcAcnY+iSyMIoebjgJfCbmHAHJ2PoksjCKHm44CXwm5hwBydj
a8Sfd2Fun55Ku50ysrrl185knMiDkfFYNHkQn6RONoF7ghkzYi84ip6f2DgiHkX2
TcL+C7j7OAjkfWWJq0P+d044EItwORdUmXRSjyhaZotpnJScFnLd0CbMAsc1mKUi
ZbUS3T10UP8HmenkJKl5PBW8MzRj2WYj/9qE4PRO0Tuh3lyVvbZDz31Zx+zxM5nW
J2Bi8Nbudw8Xw4u/oULOEeOYGQ3F0PHmWYoGSxJC0fxyRr39IYKXdjmLoNs9MiPm
L+zXM2vcsfH8j7Zz32oMHvueAuJ7e6y2D5cEIhkhO/W7SS74YBrFMcUqhZJ/MZCS
HeePHH8gBK9xB0Tp1PxpEQ9CuS46PMNJrk0/rNZEHhE6gbsCWf0KDCcvmOTUzPD8
7eTZntrHLVJDKqDeL+GQlOm9ssRt409iP0mYuNmaLIAk9FwjyHr8Fc35jEChu9hT
EcnrZCpJ+dTrAZX0i8HHtESQNL74sX6sZSP4ammyr7YmQkN2nnnYZ413O6/DnUVc
Mcp8y6p1ZzYFjl1BSw+OXGbTlceKNDdOLLmp+s87xRmJsqCNeqJy4nwkUch7eAtJ
f4sahpmSdzGfYnGYQI9bfvjuqrkaym+uFpuC7elCxAW5RaUJngG8JSZtyWefdPB+
rsmjWEXGgHBnxL9s0nVjZO/3At6PApxNehwKaAjeSvYYGaggtZeTqlUjGvbp2iz2
krUPBpyg1235d+vUQs5v0kROjqqTTrWV+f1cPuNCxCSa66DvAFcd+cbuNaaUBMua
YSBzImHsn6pZELkDF8Rl2eRw8Ke62oYL8kDGJP2qjF3oeHlCZX8RvOrn/bDqpPGX
5wA3uWLm9eGPkljhSwg2XhIXIvKvzEORBQzoqciyZVCF3gebqDyP3s95KXnv4Eui
Qqe5gaZB+ObRpWvCapXHEst15s927/mArkOoG/l/R031Q22nnXTP78mp0FQl26Fo
PMtOtk9OgAjOq7pnSXuWVNidXZWMDnumjL5yxEztXHU7605jVeliL6KxUw6zB73k
OiSP9+gALGfr4V5jgcqx3LBJJzbQxrx1uMWEaX3UyTedV7gYrH0+4zeIEMkl20Kz
7PqXMJAtsZrMcgl2jswY4hXAsMZkIluUYJNkvQmEDOliN/4FR3/SxppIOZv1L1rD
J0HK/qZChGE3+bGa1WA7OEbHaohTwmp73cUbTCodaEnW/+2vK3rBgBq6GI9Qr5bZ
FjEbwK5koxRBEaAJukPYgYlJVy+6oZVZhv3x4R3vYVNgnbj5YCvGJF9ckdfjEHBr
eBRy9+jNiXyYcjqY6H8fqhqDE+vYAlIgY0cQ1LVV5BVTp0j1EsxigEsdbq9SUQOw
Fo21sB2w6h1hFB/wvYaOIxfftYtKCCILxX5QN/JUQewJvlIsrYRfbOZ3Z+A0EGYG
trG320bf8+70PbkiW3ETYPJQD+jwIzJ1i/vu+mWJpl0bx5MtNi7GjG2cbzn19xAs
jOdbqMwvx6vwgt5zW0RacrlHmXYDJUBpNCkDSCENWYGURL1NSJg2y88ZQzEGHeit
ZZ0i5At1XzJ5n4f7mbXFF4Jzc4iRzMNOaIfOsAF3Rc7P/zGFSvmZv9zHzGMAdylB
QTfifcowZ0Cs81ZMzijDe8X+cdHqbPY5Q4jjHjgUUY2KWSTgMWxcsAaZd/X8unkn
KdxN3WbO/vazJiaBEyRidxOK2SCtetNBZeuaN34wwlk431rcervLu6hjM3/9A75L
SFjks0sXetVuPUf7mzb1FEEeWXM4dzM3DQ0vCJXQXuUOkruEhzrzG6oWwkU+kv7N
WHWEbPdZNLx4y17J/BK0vsZtGR5glFU1p8XblwEc6sAcp84ojp35T50voiKBNRn8
geTLMKBVCTYcVqoMCu3o0kKXVXjZ4OcbITETCveLTqbSRK8AOchGUP73JvBVnDkx
kiCQ3aJl4qWROezFWS/TOABQSlPrEPih4nJFUYyGyR0QYIAcrTrwxT2zFoSUsQ7F
agbiZ0QapZg+Pf9hgElqvqIaj3ENmCntjuJX2O7urzfz+aiWmk1wbdd6JVzvmegU
abVqymQScuPZzHiAi3+fSsKDLglHgVjoa63xLA2yBAAY/ih4zToa/NUXxkzmyx5U
Oc0HrM2gzyv94rX30EFN66QVc0UJBaCFvrR+rC4NdNlQiDCGwwdECeYwHxko5cib
gYxv3Dg8+evTHEEcssAdUG5DuSJya/671NP1iluHkcIGN92SfFbGyKZvmwiAcycl
rFt+NfdWJ+Na3EUIcanX/xe/f87XBF/oyamhxm9iSvgoBvxajil1j9cyyaXt9jaF
9VL9Sy7enGJXcK3m9Dxei5jNA1IsMfnsynLh4uDFgaz7OZ4aN5A760QiMgJQx3Oi
e8I71f6d7twzHcuW2WpiXtug+RjQUGXmiIQ5PAHS58VE9MoWYUC9Vg4vBmtxMiM+
3pYvB+b/glNBTgdjAqwJWgEwJoBo8Sw4rgycUOFa4E9CqN2Zdt6cEgbtPpKV98px
2ARbjptX/epuhWxmQqj6WAWbE/90cZ3iEI92eQMILtUPMeepEQXO4P3dE/QPUMAa
wfbJkt6SN1xK0o+IseHBbB5VN1qhbpYQhV0o9Yu9FOjx8bbdYuxl0NrZSsb4+WJu
R4Fx2BYUZe4buZMkrBfYP49mT3VBH3cWLD79+eCIjlR41w8/u/J90X6i4KENW44W
vHqFL64EoP14KugLSJnp7r+FNN7qcHuuIOc7EMcD0WEyREDsL0+lv4ncWa1Q6len
CT2wwy5Pk15AoeDXUBZUBX7YkIfZqBbze+jUGLQEHKdpEuLUWekUp08Uop+QCWUB
3Lsv+sw0r75aiNhvSJoTaM3BOWB9oDMWf4lJKIEgCKvnk7g4hYCQTNBeJ1LuSHVM
0J6MjQ9GWFGCe6iCe6r4DlIpdCukCQqoeKRDjgy0p6yRZlRnMmU9W/7gJ1EGZun5
81hrR12NGyAPibRmOn4LnrpeSdwaCUBbbqZk+okTTXGLMMUo3+CflMIIY2Qv0o+7
/oHWe/QyEWKvBcSOflQ+0aXaHfUoeAJefbPisuOdy4aReqQGU5I4OmWKz7cDsHfw
Rf2r0afNzgDPLMtjgy7e0O/F9QN86MwLqPLLtWSeMuJUZ4I1IIvDO8o3HBUPrnHt
EGh1JSV0P0QnkEeq0bBmTlsemXQHovEUQ9MjiqhDiaPd4cjvg6D62OMinedR3ajj
xgLYb1L97rwJM37sNknC7C6JPwnr6RMDRrwwCZhoUcfQPHg4pPyGkHS3GzQt+R7g
abjSNqmjau1HN/Nfp9ADj6OJXIMdnZ1j18fR4rwuuYmlyq0924gL3ElIhTkLWGKd
ejKssGm+mazt0+piiJkIcBnigWhzJ2yjqIDSPXuhEB2LcrGp9z3KU3pRVOfa7Pwx
Zg8wyiGPkVW+9zaUJ8m+mJfW0cMeAb6yeA1pNNUmTS1Buqn+eKlsWhAdfRd4Ni4W
xmxXlh/3zhYf9COK0/ARIh3fDeUgCA0spjdkK5BARb76G1Z7DjeIw0P+PJa3teQs
bQEjGKVixCoTKuFBWntZ1MfK0Z+f3kfip6vUwtj03g8NbO53uTvaAV2EjKK4s53e
ig/WgXQmUR72LXgkTEXtAsV5SmW2WboQccNuQn1yKbiUl/xbwRIXR8JEic1kusai
XaEUy7jYjM/fTEDdtKcoW5eW4D6WxBHzE7Ri5cAftglqfQLy/rbZdFd3l8QXbBBw
OTz1EAq+2G6hx1noLRD4xrlx6KIYxHEoDHcpWn2QS0XbYrUbf7V26VXTsKRzHPWI
T4nvEUMFJoiHzZ5T8vsOzOP1d+XFRedI7KlxbToh5+OSHrt4B2p0YEEn3I/2p/W7
CDxm0CEnH1btjEjsLigCSYvjFRwwALj/1QUoEm010UxgSN91QDMx/Dj6dDZwyj5E
rXDv6Lw8DFoK9FhW3g1LssnYqk7//nBU0uTR44+xxtlFbWbLyHBuwYLXueMuJqeJ
lx/ezg6y0vUNTNrNuTkLTe4D/yMWgG/b7s0sXUJ4pA9KPXYO7h/2UEHGyURxsROw
z7w1rRrXVY6BnMOqKORBI7As3Ng/FczNcLKL+vtHCR5/5BQ2lRDw4DLgvzJpcKNZ
+8q18Z01tbVj12xY+qSkCTHVdP29lpRb+Qc+uSMZf6ZnhUxJYWyi9zhDObRb5D9d
GeTAkcm2Y6n7RUxwyorVA2dAlCSgg/dEPUa3v3krYHJ0tyTrBPEzvtlIWO+3trO4
xgWPiQ7uPI4r9G1yednwANzYY3uE58PWiFR3zKFlc/A9MFFYGQXxU720Qu6PGK3i
veqCFmxw34aFu010S/ms7SH8A6v4uQ7zRpqJUH5YJmIsaPr6+8niJMKND3299Mrh
xbMo5vmMm5bmcIGb3a/C900khnkFEFPrjim4zHobrN+F2dVPB+pbK2qXhXams3jp
zuWl5CpvhLWDDklFhDJ5ibhse4cpAAK9q3slnQop5G5EB6l85Ghlci+ZQespIxpa
Euk874xDGue/38Pg34RgvM4tQVJGCBN9MFQZ7EyROpZpBADiDEp1C8wlpAYRg23c
ZbrHEaLx01RUZjdATwHUymM0IwuIGFDxceZ+rJ1hT7BdbqraKFggo3wdOrtO+2H+
ELu5VXDknsPUepK0oCYcb6RXOTxyOwJ+9Ja2L1gSH8qScBbVbYhzf6AfMZx9zH1E
3o5w2hwBMLO0E6tLjXEDkuxtX4ukQrRa76MYw0PAY067YhAKcFtrdjrUeH+V0RX4
ERThWTxjiZGq6sM9pJQz9lG167ji8XwcdcO0JEgChUJSyvtb4H2sA4oyi56bX5ee
Ou7WRUeLz5BS9ZWKAcl735BtMxO64LafeZeDR7pnUT349CUnExMwGFP2kxz+zQFk
3o5w2hwBMLO0E6tLjXEDknIEoYqQTRwFpEmjr6cLQqwBwqU1H4ZxoSdd4tYZMiC5
YI4bFJQEfIPwExTIDsrKv6Nnsrrh/KiqMBkgBe7ND1Z4FHL36M2JfJhyOpjofx+q
uUzslyprUm+CfLJmppZ5B7sDsLKKkRUgmg0ZNRUmthiX6xIMkMFrvdkSRr+LZZX8
Ez/Tl+xflJTZ0m9/V1S7O5vwXRPi9Qzvtc3KOCvr1bBGsituk72wZ5s3UqDzVHnc
xAhnQkxU16w2D3ogv22zPvy4O8vjNUzqBxZrkSQb46ZnSzR1nkJ5gCZFiNwxSnmt
T+yZ8gG3TLzqJqoiMTkgmsMwhkS3Im/FQGgsbU5vo0B/7WvOz/nHg6sWjV3DhWio
k1pJa7vPQV+AcuTFFlns+nRFXBRu8uzWrkpLa9rRQXfhNhSvq+mO+Cpas09ReVcF
3y+swG2/b3Z+adV9X33MdYDng3XOnvBrgvzDT3eEaWNhuCMIGz1qVhOw85b73NBN
MCFopbWQ5T8AFPzKfNAYy4tWV2Khty+Y83tKli7lhFzGrFqqbjfMPPUAcSL1SERT
8aWblsMYuP5dG6UIJDaPaDQFBZuhLOCNZPNskyDFAwWL9A/niHezueB8DzHQOX5W
WvCttM7eu3ztLeQh3GtLdpeWBolVdMa8inWHWVrVNOPArxmWpbGmeA8TiqQCEWoT
gHcI6Z00oZKcdk/LyO7pPsLcmqEksjcravoMvjVAcLB4xIR7HRorUYDmunOjehsr
W8WOQiaAvBmqBVTHJJBzCySnnlIxO+qi2+AMjBUbJIUGN92SfFbGyKZvmwiAcycl
PjXvoxV7byyB3Rq1mSsdHRrlLgMEuaj8GOg4ABjNPs5gfm2kqOLTM0s2yKjNoaDx
Rf1yRZ/FSg+3RhQQ/UJMf6B7seQWZpJfqNRhtqUrYVailvpsB9SkuUfYbhkWeGAQ
k4A5HdiOG7zQ5T/4K/gWcFLSdwiDAZV7gfel0uuoLe5C7m5am2Qw9RCuAdCzvO65
M8ncv6TQ4l+yWBoPIXHBF7l0GNxMxCa8XvRcVQWukB2dNDedO1lkCV2xUW17Dag+
R47nJ9d0OYMrZh3NbjUnqpO0hI0ezs9rVuEslwb5kp2J33ogome6FF47uiBSy+nI
qC2q/qDln+AchnWfIkupeRDqnf9f0ZDxuFrU0Yb19nbRYxnzs6W9LTc/quhyoQKN
P1JOfOOqhfiu7WG+DRtSoLpDKAcXYWStW37hKKLoZ+gnOTbPKdnGcEVbdhapfCQS
5dlozHuKpDhIkWQ1XlWYW8cl8SgDG0hvqLLut04oSqmRalqM/P+9P78l4XoPcXtI
AFBKU+sQ+KHickVRjIbJHYlf2KUo03OmTpkW3sAOLug1R1EhkT1wz7IG+oj385yS
uAjDwxfprbc+jBj06GQLr+wDgMDk2x/zJtDrsVd3ZdGnO/ts7EwwgitrYHZkP4ZS
5m8UPjUNxUlYxQtnMO0hTeUeiELF09zGdncEYwezfxQhjrpKxH6fWjEtSp1sxSYG
1EkUP56BiUJQSowqFmXGqFYHKMDcCU5gjcr3rB/N5+70B4EmXvdwG62acRSDRaNr
zNsdmpMfI5SzAWfoOe8n2/EtFpYhjR4a3mUpQ9Dbtzd/GB07soDXvwa5GK+o+2OD
t102M3RThC4+KcTDKvYSoEq4ULiFbJNKHw3lf3nE4XpNKIW3pBbOSK9kMQafz8I9
83/ldVHSr/H1WF85GucZ3SLGzSvIkMA88jU12y7ntXRYsiOry7UkkZqsmLQP+utf
aPFTS/N3yuwJjXOUGJlOcZLN+lYTAloT60HCJjhydT20G0JyU7ny3W0pnL53NYZu
UKIWmlCn5ma1ZlTnNqwPe6i/7v1RyiIshEiCcNIQ4j6RuD3D9MfxWB1H6/cOQeAM
qTNgmmXaj1EVEpjvhEHtZ2yE/SfAiqkh2KzefyAbHfvmbxQ+NQ3FSVjFC2cw7SFN
XgQFLCD09GQcOro+2CNfubqGzUzu4zaFD0yQqPf6ue4dr9UwJV1qiMRYAdbnsPeS
gBPpst5RNgabQ7sdRWwEep9NDRw2bIlMYI35ZwEgERh0j1WpEVpkaqVLBf8O6omi
YYdakZw40R+yduL/QvCAMsTQDqb23irRgVgu1SDQ3wACQNrUew8NA+OxF4TyU4eE
